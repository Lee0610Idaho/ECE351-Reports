%%%%%%%%%%%%%%%%%%%%%%%%%%%%%%%%%%%%%%%%%%%
%					                      %
% Jackie Lee				              %
% ECE 351 Section 51			          %
% Lab 2					                  %
% Due February 1 2022			          %
% Latex Code for Lab 2 Report		      %
% Utilized Template on Canvas		      %
%%%%%%%%%%%%%%%%%%%%%%%%%%%%%%%%%%%%%%%%%%%
%%%%%%%%%%%%%%%%%%%%%%%%%%%%%%%%%%%%%%%%%%%
%%% DOCUMENT PREAMBLE %%%
\documentclass[12pt]{report}
\usepackage[english]{babel}
%\usepackage{natbib}
\usepackage{url}
\usepackage[utf8x]{inputenc}
\usepackage{amsmath}
\usepackage{graphicx}
\graphicspath{{images/}}
\usepackage{parskip}
\usepackage{fancyhdr}
\usepackage{vmargin}
\usepackage{listings}
\usepackage{hyperref}
\usepackage{xcolor}
\definecolor{codegreen}{rgb}{0,0.6,0}
\definecolor{codegray}{rgb}{0.5,0.5,0.5}
\definecolor{codeblue}{rgb}{0,0,0.95}
\definecolor{backcolour}{rgb}{0.95,0.95,0.92}
\lstdefinestyle{mystyle}{
backgroundcolor=\color{backcolour},
commentstyle=\color{codegreen},
keywordstyle=\color{codeblue},
numberstyle=\tiny\color{codegray},
stringstyle=\color{codegreen},
basicstyle=\ttfamily\footnotesize,
breakatwhitespace=false,
breaklines=true,
captionpos=b,
keepspaces=true,
numbers=left,
numbersep=5pt,
showspaces=false,
showstringspaces=false,
showtabs=false,
tabsize=2
}
\lstset{style=mystyle}
\setmarginsrb{3 cm}{2.5 cm}{3 cm}{2.5 cm}{1 cm}{1.5 cm}{1 cm}{1.5 cm}
\title{Lab 2: User Defined Functions}
% Title
\author{ Jackie lee}
% Author
\date{February 1 2022}
% Date
\makeatletter
\let\thetitle\@title
\let\theauthor\@author
\let\thedate\@date
\makeatother
\pagestyle{fancy}
\fancyhf{}
\rhead{\theauthor}
\lhead{\thetitle}
\cfoot{\thepage}
%%%%%%%%%%%%%%%%%%%%%%%%%%%%%%%%%%%%%%%%%%%%
\begin{document}
%%%%%%%%%%%%%%%%%%%%%%%%%%%%%%%%%%%%%%%%%%%%%%%%%%%%%%%%%%%%%%%%%%%%%%%%%%
%%%%%%%%%%%%%%%
\begin{titlepage}
\centering
\vspace*{0.5 cm}
% \includegraphics[scale = 0.075]{bsulogo.png}\\[1.0 cm] % 
University of Idaho
\begin{center}    \textsc{\Large   ECE 351 - Section 51 \ }\\[2.0 cm]
\end{center}% University Name
\textsc{\Large February 1 2022  }\\[0.5 cm] % Course 
\rule{\linewidth}{0.2 mm} \\[0.4 cm]
{ \huge \bfseries \thetitle}\\
\rule{\linewidth}{0.2 mm} \\[1.5 cm]
\begin{minipage}{0.4\textwidth}
\begin{flushleft} \large
% \emph{Submitted To:}\\
% Name\\
% Affiliation\\
%contact info\\
\end{flushleft}
\end{minipage}~
\begin{minipage}{0.4\textwidth}
\begin{flushright} \large
\emph{Submitted By :} \\
Jackie Lee
\end{flushright}
\end{minipage}\\[2 cm]
% \includegraphics[scale = 0.5]{PICMathLogo.png}
\end{titlepage}
%%%%%%%%%%%%%%%%%%%%%%%%%%%%%%%%%%%%%%%%%%%%%%%%%%%%%%%%%%%%%%%%%%%%%%%%%%
%%%%%%%%%%%%%%%
\tableofcontents
\pagebreak
%%%%%%%%%%%%%%%%%%%%%%%%%%%%%%%%%%%%%%%%%%%%%%%%%%%%%%%%%%%%%%%%%%%%%%%%%%
%%%%%%%%%%%%%%%
\renewcommand{\thesection}{\arabic{section}}
\section{Introduction}
In this lab, the purpose was to gain experience with creating and plotting functions 
in Python. Many of the functions were already introduced in class such as the cosine and step functions
so the expected plots and equations were known. This lab shows us how easy it is to have multiple plots
compared to doing them by hand. 
\section{Equations}
The functions used in this lab were familiar due to working with them in class. Equation 5 is a combination of
the step function and ramp functions which are described 1-4. 
\begin{equation*}
y = cos(t)
\end{equation*}
\begin{equation}
u(t) = 0 | t < 0
\end{equation}
\begin{equation}
u(t) = 1 | t \geq 0
\end{equation}
\begin{equation}
r(t) = 1 | t >= 0
\end{equation}
\begin{equation}
r(t) = 1 | t >= 0
\end{equation}
\begin{equation}
    y (t)=  r(t) - r(t - 3) + 5 * u(t- 3) - 2 * u(t - 6) - 2 * r(t - 6) 
\end{equation}
\section{Methodology}
Most of my process was going over the example code that was in the lab handout. I took what was necessary
and modified it so it fit what I needed for the step function, ramp function, and the cosine function. Figuring out 
to get the correct equation for the figure in the lab handout took longer and I had to reference the textbook for examples
of deriving an equation from many signals. Once I understood the textbook examples, creating the equation took some trial and 
error to get it right. Using Spyder makes it time efficient to do guess and check and figure out what needs to be changed in the 
equation. Graphing the derivative was based off knowing the slopes at each interval where the slope changes. For the most part, it
consisted of a bunch of plotting that used the same reference code for a plot but modified either the axis, title, or function. 
\begin{lstlisting}[language=Python]
#%% Example Code Plotting the Cosine Function
plt.figure(figsize = (10,7))
plt.subplot(2, 1, 1)
plt.plot(t, y)
plt.grid()
plt.ylabel('y(t)')
plt.xlabel('t')
plt.title('Cosine(t)')
\end{lstlisting}
\section{Results}
Overall it went pretty smoothly, I knew what the functions were supposed to look like
so I was able to verify plots. Only one that was difficult was the derivative plot 
because I was not positive on how an infinite slope would be represented. Although I had 
a rough idea based on getting the slopes from the ramp functions and was told by the lab
handout that they wouldn't match what I had drawn anyways. The resulting plots including the hand drawn derivative plot
are presented at the end of the report.
\section{Error Analysis}
I did not have any huge difficulties with the lab, it would have been beneficial to have the exact derivative plot but the 
output that was given seemed to match expectations. If it was a test I would have messed up on the equation for the figure because
I took advantage of the ability for Spyder to produce plots effortlessly to check my equations. Reviewing the material in the textbook
helped me understand the concepts of the lab better and the example code helped with coding. 
\section{Questions}
1.The plots from Part 3 Task 4 and Part 3 Task 5 are not identical, the problem with my hand drawn one is 
I thought the changes in slope were instant but Spyder shows that it is gradual. I wouldn't say it's possible for this figure
to have the same drawing because the slopes of the plot from Spyder are not constant so it would be difficult to replicate by hand.\\
2. The correlation gets better as the step size increases because the step size indicates the resolution of the plot. With a worse resolution the correlation between the two plots will get better since the hand drawn plot has a poor resolution compared to Spyder's.\\
3. I think the only question if I have to put the textbook as reference in the Bibliography or if the Bibliography section should always
be necessary. I think my biggest issue with Latex this time around is that I could not get the images(plots) to be between two sections (Results and Error Analysis). A few of them would but the rest of them would go after Error Analysis so the plots would not be together.
\section{Conclusion}
I learned more about functions in Python and the useful plotting capabilities it has compared to C and C++. The function implementation 
was a bit different than what I was expecting but using the example code helped me generalize for use with other functions in this lab. I would have liked to learn more about the derivative function and its plot since it was not really talked about in lab. It felt more like how to use the function in Python. It felt a bit cumbersome having so many plots, I would recommend only doing one shift and one scale since it seemed repetitive. Perhaps defining the time scale would have also been helpful. Overall this lab was a great success working with Python and developing an equation for a given plot. 
\newpage
\section{Plots}
\begin{figure}[ht]
\includegraphics[width=5in]{Lab 2 Part 1 Task 2 Plot.png}
\end{figure}
\begin{figure}[ht]
\includegraphics[width=5in]{Ramp Plot.png}
\end{figure}
\begin{figure}[ht]
\includegraphics[width=5in]{Step Plot.png}
\end{figure}
\begin{figure}[ht]
\includegraphics[width=5in]{Fig 2 Plot.png}
\end{figure}

\pagebreak
\begin{figure}[ht]
\includegraphics[width=5in]{Time Reversal Plot.png}
\end{figure}
\begin{figure}[ht]
\includegraphics[width=5in]{Shifted Right 4 Plot.png}
\end{figure}
\pagebreak
\begin{figure}[ht]
\includegraphics[width=5in]{Fliped and Shifted 4 Plot.png}
\end{figure}
\begin{figure}[ht]
\includegraphics[width=5in]{Scaled by Double Plot.png}
\end{figure}
\begin{figure}[ht]
\includegraphics[width=5in]{Scaled by Half Plot.png}
\end{figure}
\begin{figure}[ht]
\includegraphics[width=5in]{lab2.png}
\caption{Hand Drawn Derivative of Figure 2}
\end{figure}
\begin{figure}[ht]
\includegraphics[width=5in]{Deriviative Plot.png}
\end{figure}
\pagebreak

\newpage

\newpage
\begin{thebibliography}{111}
\bibitem{ACMT}
Dennis M. Sullivan,
Signals and Systems for Electrical Engineers I,
 2018
\end{thebibliography}
\end{document}