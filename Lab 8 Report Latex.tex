%%%%%%%%%%%%%%%%%%%%%%%%%%%%%%%%%%%%%%%%%%%
%					                      %
% Jackie Lee				              %
% ECE 351 Section 51			          %
% Lab 8					                  %
% Due March 22 2022			              %
% Latex Code for Lab 8 Report		      %
% Utilized Template on Canvas		      %
%%%%%%%%%%%%%%%%%%%%%%%%%%%%%%%%%%%%%%%%%%%

%%%%%%%%%%%%%%%%%%%%%%%%%%%%%%%%%%%%%%%%%%%
%%% DOCUMENT PREAMBLE %%%
\documentclass[12pt]{report}
\usepackage[english]{babel}
%\usepackage{natbib}
\usepackage{url}
\usepackage[utf8x]{inputenc}
\usepackage{amsmath}
\usepackage{graphicx}
\graphicspath{{images/}}
\usepackage{parskip}
\usepackage{fancyhdr}
\usepackage{vmargin}
\usepackage{listings}
\usepackage{hyperref}
\usepackage{xcolor}

\definecolor{codegreen}{rgb}{0,0.6,0}
\definecolor{codegray}{rgb}{0.5,0.5,0.5}
\definecolor{codeblue}{rgb}{0,0,0.95}
\definecolor{backcolour}{rgb}{0.95,0.95,0.92}

\lstdefinestyle{mystyle}{
    backgroundcolor=\color{backcolour},   
    commentstyle=\color{codegreen},
    keywordstyle=\color{codeblue},
    numberstyle=\tiny\color{codegray},
    stringstyle=\color{codegreen},
    basicstyle=\ttfamily\footnotesize,
    breakatwhitespace=false,         
    breaklines=true,                 
    captionpos=b,                    
    keepspaces=true,                 
    numbers=left,                    
    numbersep=5pt,                  
    showspaces=false,                
    showstringspaces=false,
    showtabs=false,                  
    tabsize=2
}
 
\lstset{style=mystyle}

\setmarginsrb{3 cm}{2.5 cm}{3 cm}{2.5 cm}{1 cm}{1.5 cm}{1 cm}{1.5 cm}

\title{Lab 8: Fourier Series Approximation of a Square Wave}								
% Title
\author{ Jackie Lee}						
% Author
\date{March 22}
% Date

\makeatletter
\let\thetitle\@title
\let\theauthor\@author
\let\thedate\@date
\makeatother

\pagestyle{fancy}
\fancyhf{}
\rhead{\theauthor}
\lhead{\thetitle}
\cfoot{\thepage}
%%%%%%%%%%%%%%%%%%%%%%%%%%%%%%%%%%%%%%%%%%%%
\begin{document}

%%%%%%%%%%%%%%%%%%%%%%%%%%%%%%%%%%%%%%%%%%%%%%%%%%%%%%%%%%%%%%%%%%%%%%%%%%%%%%%%%%%%%%%%%

\begin{titlepage}
	\centering
    \vspace*{0.5 cm}
   % \includegraphics[scale = 0.075]{bsulogo.png}\\[1.0 cm]	% University Logo
\begin{center}    \textsc{\Large   ECE 351 - Section 51 }\\[2.0 cm]	\end{center}% University Name
	\textsc{\Large March 8 2022  }\\[0.5 cm]				% Course Code
	\rule{\linewidth}{0.2 mm} \\[0.4 cm]
	{ \huge \bfseries \thetitle}\\
	\rule{\linewidth}{0.2 mm} \\[1.5 cm]
	
	\begin{minipage}{0.4\textwidth}
		\begin{flushleft} \large
		%	\emph{Submitted To:}\\
		%	Name\\
          % Affiliation\\
           %contact info\\
			\end{flushleft}
			\end{minipage}~
			\begin{minipage}{0.4\textwidth}
            
			\begin{flushright} \large
			\emph{Submitted By :} \\
			Jackie Lee  
		\end{flushright}
           
	\end{minipage}\\[2 cm]
	
%	\includegraphics[scale = 0.5]{PICMathLogo.png}
    
    
    
    
	
\end{titlepage}

%%%%%%%%%%%%%%%%%%%%%%%%%%%%%%%%%%%%%%%%%%%%%%%%%%%%%%%%%%%%%%%%%%%%%%%%%%%%%%%%%%%%%%%%%

\tableofcontents
\pagebreak

%%%%%%%%%%%%%%%%%%%%%%%%%%%%%%%%%%%%%%%%%%%%%%%%%%%%%%%%%%%%%%%%%%%%%%%%%%%%%%%%%%%%%%%%%
\renewcommand{\thesection}{\arabic{section}}
\section{Introduction}
 

In this lab we get experience using Fourier series. We were able to utilze Python to plot the Fourier series for different values of N and calculate the coefficients. 

\section{Equations}
Part 1:
\begin{equation*}
a_0 = 0
\end{equation*}

\begin{equation*}
a_k = 0
\end{equation*}


\begin{equation*}
b_k = 2[\frac{1-cos(\pi k)}{\pi k}] 
\end{equation*}

\begin{equation*}
x(t) = \sum_{n=1}^{\infty} [\frac{2(1-cos(\pi k))sin(kwt)}{\pi k}]
\end{equation*}


\section{Methodology}

Finding the values for the coefficients was straightforward as it only required substituting different values of k in the equations that were solved for during the prelab. Doing a summation required using a for loop that would then store its previous value using +=. 

\begin{lstlisting}[language=Python]
#%% Code that provided the value of the coefficients for k = 1, similar format with k = 2 and k = 3. Did not supply the a coefficient equation since it did not rely on k so it would have been redundant to have it calculate it for each k value. 

## k = 1 ##
print("k=1")
k = 1
bk = 2*((1-np.cos(np.pi * k))/(np.pi * k))
print(bk)
print(ak)

\end{lstlisting}

\begin{lstlisting}[language=Python]
#%% Code that implemented the fourier sreies for N = 1 using a for loop and a and add assignment operator. This is repeated for different N values with the same format. Starts at 1 due to the first term of the Fourier series being at one thus in order for N to be 1 it requires adding 1. 

for k in range(1,1+1): #+1 due to starting at 1 based on the given series
    bk = 2*((1-np.cos(np.pi * k))/(np.pi * k))
    s1 += bk*np.sin(k*w*t) #svalue for a specified N

\end{lstlisting}
\section{Results}

The results matched what I was expecting when I computed the values manually with my calculator. The plots resulted in the same shape as the plot in the handout. We can tell that as N gets higher, the better the resulting plot would be although it has diminishing returns as seen between 150 and 1500. 
\\
Printed Output:
\\ \includegraphics[scale = 1.5 ]{Results.png}

N = 1, 3, 15:
\\ \includegraphics[scale=0.6]{N 1-15.png}
\newpage
N = 50, 150, 1500:
\\ \includegraphics[scale=0.6]{N 50-1500.png}




\section{Error Analysis}

The difficult part was solving the prelab. It was rough to keep everything on Latex so a lot of the basic steps were left out which could have led to errors. However, when lab had started being able to verify the solution helped me get started with the lab where I could have been juggling whether my code was wrong or my equation. 

\section{Questions}
1. We can tell that x(t) is an even function because it is symmetric across the y axis. 

2. Since the values of a is not dependent on k, we can expect it be the same for all values (1-n) hence it would be 0. 

3. We can tell that the square wave improves as N reaches higher values where the best N value in terms of shape and number of computations is 50.  It could be argued for N = 15that it does not resemble a square wave where N = 3 and N = 1 are closer to trig wave forms than square wave forms. 

4. When N increases, the summation adds more values utilizing more waveforms to create a more accurate representation. 

5. No feedback, lab was smooth after clarification over the prelab. 


\section{Conclusion}

By the end of the lab I was able to utilize Python to verify my coefficients of a Fourier series and see how the value N impacts the representation of a square wave using sine and cosine waveforms by plotting them. This would only be feasible electronically as manually calculating the values of the summations for the respective N values would be unreasonable. 

\newpage


\begin{thebibliography}{111}
\bibitem{ACMT}
Dennis M. Sullivan,
Signals and Systems for Electrical Engineers I,
 2018
\end{thebibliography}
\end{document}
