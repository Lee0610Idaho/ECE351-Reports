%%%%%%%%%%%%%%%%%%%%%%%%%%%%%%%%%%%%%%%%%%%
%					                      %
% Jackie Lee				              %
% ECE 351 Section 51			          %
% Lab 7					                  %
% Due March 8 2022			              %
% Latex Code for Lab 7 Report		      %
% Utilized Template on Canvas		      %
%%%%%%%%%%%%%%%%%%%%%%%%%%%%%%%%%%%%%%%%%%%

%%%%%%%%%%%%%%%%%%%%%%%%%%%%%%%%%%%%%%%%%%%
%%% DOCUMENT PREAMBLE %%%
\documentclass[12pt]{report}
\usepackage[english]{babel}
%\usepackage{natbib}
\usepackage{url}
\usepackage[utf8x]{inputenc}
\usepackage{amsmath}
\usepackage{graphicx}
\graphicspath{{images/}}
\usepackage{parskip}
\usepackage{fancyhdr}
\usepackage{vmargin}
\usepackage{listings}
\usepackage{hyperref}
\usepackage{xcolor}

\definecolor{codegreen}{rgb}{0,0.6,0}
\definecolor{codegray}{rgb}{0.5,0.5,0.5}
\definecolor{codeblue}{rgb}{0,0,0.95}
\definecolor{backcolour}{rgb}{0.95,0.95,0.92}

\lstdefinestyle{mystyle}{
    backgroundcolor=\color{backcolour},   
    commentstyle=\color{codegreen},
    keywordstyle=\color{codeblue},
    numberstyle=\tiny\color{codegray},
    stringstyle=\color{codegreen},
    basicstyle=\ttfamily\footnotesize,
    breakatwhitespace=false,         
    breaklines=true,                 
    captionpos=b,                    
    keepspaces=true,                 
    numbers=left,                    
    numbersep=5pt,                  
    showspaces=false,                
    showstringspaces=false,
    showtabs=false,                  
    tabsize=2
}
 
\lstset{style=mystyle}

\setmarginsrb{3 cm}{2.5 cm}{3 cm}{2.5 cm}{1 cm}{1.5 cm}{1 cm}{1.5 cm}

\title{Lab 7: Block Diagrams and System Stability}								
% Title
\author{ Jackie Lee}						
% Author
\date{March 8}
% Date

\makeatletter
\let\thetitle\@title
\let\theauthor\@author
\let\thedate\@date
\makeatother

\pagestyle{fancy}
\fancyhf{}
\rhead{\theauthor}
\lhead{\thetitle}
\cfoot{\thepage}
%%%%%%%%%%%%%%%%%%%%%%%%%%%%%%%%%%%%%%%%%%%%
\begin{document}

%%%%%%%%%%%%%%%%%%%%%%%%%%%%%%%%%%%%%%%%%%%%%%%%%%%%%%%%%%%%%%%%%%%%%%%%%%%%%%%%%%%%%%%%%

\begin{titlepage}
	\centering
    \vspace*{0.5 cm}
   % \includegraphics[scale = 0.075]{bsulogo.png}\\[1.0 cm]	% University Logo
\begin{center}    \textsc{\Large   ECE 351 - Section 51 }\\[2.0 cm]	\end{center}% University Name
	\textsc{\Large March 8 2022  }\\[0.5 cm]				% Course Code
	\rule{\linewidth}{0.2 mm} \\[0.4 cm]
	{ \huge \bfseries \thetitle}\\
	\rule{\linewidth}{0.2 mm} \\[1.5 cm]
	
	\begin{minipage}{0.4\textwidth}
		\begin{flushleft} \large
		%	\emph{Submitted To:}\\
		%	Name\\
          % Affiliation\\
           %contact info\\
			\end{flushleft}
			\end{minipage}~
			\begin{minipage}{0.4\textwidth}
            
			\begin{flushright} \large
			\emph{Submitted By :} \\
			Jackie Lee  
		\end{flushright}
           
	\end{minipage}\\[2 cm]
	
%	\includegraphics[scale = 0.5]{PICMathLogo.png}
    
    
    
    
	
\end{titlepage}

%%%%%%%%%%%%%%%%%%%%%%%%%%%%%%%%%%%%%%%%%%%%%%%%%%%%%%%%%%%%%%%%%%%%%%%%%%%%%%%%%%%%%%%%%

\tableofcontents
\pagebreak

%%%%%%%%%%%%%%%%%%%%%%%%%%%%%%%%%%%%%%%%%%%%%%%%%%%%%%%%%%%%%%%%%%%%%%%%%%%%%%%%%%%%%%%%%
\renewcommand{\thesection}{\arabic{section}}
\section{Introduction}
 

In this lab we get experience with block diagrams, more importantly knowing the differences between closed loop an open loop along with poles and zeros. Another part was to be able to determine the stability of a system by hand. We are able to check our work for stability and factorization k using Python's tf2zpk function and plotting capabilities. 

\section{Equations}
Part 1:
\begin{equation*}
G(s) = \frac{s+9}{(s^2-6s-16)(s+4)} = \frac{s+9}{(s-8)(s+2)(s+4)}
\end{equation*}
Poles(s = 8, -2, -4) Zeros (s = -9)

\begin{equation*}
A(s) = \frac{s+4}{s^2+4s+3} = \frac{s+4}{(s+3)(s+1)} 
\end{equation*}
Poles(s = -3, -1) Zeros (s = -4)

\begin{equation*}
B(s)= s^2+26s+168 = (s+14)(s+12)
\end{equation*}
Poles(None) Zeros(S = -12, -14)

Open Loop Transfer Function:
\begin{equation*}
H(s) = \frac{s+9}{(s-8)(s+2)(s+3)(s+1)}
\end{equation*}

Part 2:
Closed Loop Transfer Function:
\begin{equation*}
H(s) = \frac{numG*numA}{denA*denG + denA*numB*numG}
\end{equation*}
\begin{equation*}
H(s) = \frac{(s+1)(s+13)(s+36)}{(s+2)(s+41)(s+500)(s+2995)(s+6878)(s+4344)}
\end{equation*}



\section{Methodology}

For Part 1, I was able to factor it by hand and using a calculator to check my work beforehand. The poles and zeros were a matter of setting each factor to zero and solving for s. I wasn't sure about the open loop transfer function until it was brought up by the TA in lab. Afterwards it was clear what it meant in terms of the diagram. I utilized Python to check my work and to ensure whether or not the open loop was stable. 

\begin{lstlisting}[language=Python]
#%% Code that sig.tfzpk() and np.roots() to print the zeros, poles, and gain of the functions and sig.convolve to plot the step response of the open loop transfer function. 

zeros, poles, gain = sig.tf2zpk([1,9], sig.convolve([1,-6,-16],[1,4]))
print("G(s):")
print(zeros)
print(poles)
print(gain)

zeros, poles, gain = sig.tf2zpk([1,4], [1,4,3])
print("A(s):")
print(zeros)
print(poles)
print(gain)

roots = np.roots([1,26,168])
print("B(s):")
print(roots)
print()

system_open = [1,9],sig.convolve([1,4,3],[1,-6,-16])
print(sig.convolve([1,4,3],[1,-6,-16]))
t, step = sig.step(system_open)

\end{lstlisting}


Part 2 was more familiar because closed loops were what we have been dealing with in lecture. It took a bit to get the correct equation in symbolic form but was able to reach an equation in the end that I checked with others. Similarly to part 1, I was able to plot and print my results to get the arithmetic values and check the stability of the closed loop transfer function. 

\begin{lstlisting}[language=Python]
#%% Code that implemented the numerators and denominators of each equation symbolically to use the sig.convolve() to find the resulting transfer function for closed loop and its step response using sig.step(). 

numG = [1,9]
denG = sig.convolve([1,-6,-16],[1,4])
numA = [1,4]
denA = [1,4,3]
numB = [1,26,168]
denB = [1]


numTotal = sig.convolve(numG, numA)
denTotal = sig.convolve(denA, denG) + sig.convolve(sig.convolve(denA, numB), numG)

print(numTotal)
print(denTotal)

transfer_closed = numTotal, denTotal
t, step_closed = sig.step(transfer_closed)
\end{lstlisting}
\section{Results}

Overall, the plots were what I expected in terms of the shape. It was straightforward on whether or not the system was stable based on if the plot reached a determinate value as time approaches infinity. This also seemed to match the printed results from Python. 
\\ \includegraphics[scale=0.6]{Open.png}

\\ \includegraphics[scale=0.6]{Closed.png}

Part 1 Printed Output
\\ \includegraphics[scale = 1.0 ]{Results.png}



\section{Error Analysis}

The difficult part was understanding the differences between open loop and closed loop since it wasn't really discussed in class. This made me unsure if I handled part 1 correctly despite the handout saying that the open loop had no feedback. However, it was later discussed in lab and from there was able to verify my understanding. 

\section{Questions}
1. We can tell that the open loop transfer function found in task 3 in part 1 in unstable because we have a pole that is positive (s = 8). This is supported by its step response plot where its value goes to indefinite as time approaches infinity having an indeterminate value.

2. Similarly to the open loop from part 1, the closed loop from task 3 in part 2 is stable because it has no positive poles (all negative). It's resulting plot from task 4 clearly shows that it settles at a value around .008 unlike the open loop's step response. 

3. scipy.signal.convolve() using factored terms results in the expanded form because the convolution of two functions in the time domain is the equivalent to multiplication of the two functions in the Laplace domain.This wouldn't work in lab 3 because it would require more than to just multiply them. 

4. The open loop system utilizes the quickest path from input to output so it discards paths while closed loop utilizes all paths from x to y. We can tell that although open loop system was easier it is unstable compared to the closed loop system. 

5. Scipy.signal.residue() was used for partial fraction expansion resulting in residues and poles of the function while scipy.signal.tf2zpk does not perform partial fraction expansion, only returning the zeroes, poles, and the gain of a linear function.

6. Because the stability of a system is based on whether or not its poles are positive/negative. I would say it's possible for both cases because if the equations were different or forcibly meant to have positive/negative poles then the open/closed loop would have their stability changed. It's easier to tell with the open loop since it's just the multiplication between A(S) and G(s) but this should apply to the closed loop transfer function as well. I would be surprised if it wasn't possible since it seems unrealistic otherwise. 

7. No feedback, everything went smooth after the explanation in lab. 

\section{Conclusion}

By the end of the lab I was able to utilize Python's functions to determine the poles and zeros of a function while also plotting the step response of a closed loop and open loop system from a block diagram. I was able to learn about whether or not a system was stable by looking at its equation and its plot and learn the differences between closed loop and open loop systems which were not discussed in lecture. It was overall a great lab. 

\newpage


\begin{thebibliography}{111}
\bibitem{ACMT}
Dennis M. Sullivan,
Signals and Systems for Electrical Engineers I,
 2018
\end{thebibliography}
\end{document}
