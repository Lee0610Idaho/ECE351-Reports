%%%%%%%%%%%%%%%%%%%%%%%%%%%%%%%%%%%%%%%%%%%
%					                      %
% Jackie Lee				              %
% ECE 351 Section 51			          %
% Lab 6					                  %
% Due March 1 2022			              %
% Latex Code for Lab 6 Report		      %
% Utilized Template on Canvas		      %
%%%%%%%%%%%%%%%%%%%%%%%%%%%%%%%%%%%%%%%%%%%

%%%%%%%%%%%%%%%%%%%%%%%%%%%%%%%%%%%%%%%%%%%
%%% DOCUMENT PREAMBLE %%%
\documentclass[12pt]{report}
\usepackage[english]{babel}
%\usepackage{natbib}
\usepackage{url}
\usepackage[utf8x]{inputenc}
\usepackage{amsmath}
\usepackage{graphicx}
\graphicspath{{images/}}
\usepackage{parskip}
\usepackage{fancyhdr}
\usepackage{vmargin}
\usepackage{listings}
\usepackage{hyperref}
\usepackage{xcolor}

\definecolor{codegreen}{rgb}{0,0.6,0}
\definecolor{codegray}{rgb}{0.5,0.5,0.5}
\definecolor{codeblue}{rgb}{0,0,0.95}
\definecolor{backcolour}{rgb}{0.95,0.95,0.92}

\lstdefinestyle{mystyle}{
    backgroundcolor=\color{backcolour},   
    commentstyle=\color{codegreen},
    keywordstyle=\color{codeblue},
    numberstyle=\tiny\color{codegray},
    stringstyle=\color{codegreen},
    basicstyle=\ttfamily\footnotesize,
    breakatwhitespace=false,         
    breaklines=true,                 
    captionpos=b,                    
    keepspaces=true,                 
    numbers=left,                    
    numbersep=5pt,                  
    showspaces=false,                
    showstringspaces=false,
    showtabs=false,                  
    tabsize=2
}
 
\lstset{style=mystyle}

\setmarginsrb{3 cm}{2.5 cm}{3 cm}{2.5 cm}{1 cm}{1.5 cm}{1 cm}{1.5 cm}

\title{Lab 6: Partial Fraction Expansion}								
% Title
\author{ Jackie Lee}						
% Author
\date{March 1}
% Date

\makeatletter
\let\thetitle\@title
\let\theauthor\@author
\let\thedate\@date
\makeatother

\pagestyle{fancy}
\fancyhf{}
\rhead{\theauthor}
\lhead{\thetitle}
\cfoot{\thepage}
%%%%%%%%%%%%%%%%%%%%%%%%%%%%%%%%%%%%%%%%%%%%
\begin{document}

%%%%%%%%%%%%%%%%%%%%%%%%%%%%%%%%%%%%%%%%%%%%%%%%%%%%%%%%%%%%%%%%%%%%%%%%%%%%%%%%%%%%%%%%%

\begin{titlepage}
	\centering
    \vspace*{0.5 cm}
   % \includegraphics[scale = 0.075]{bsulogo.png}\\[1.0 cm]	% University Logo
\begin{center}    \textsc{\Large   ECE 351 - Section 51 }\\[2.0 cm]	\end{center}% University Name
	\textsc{\Large March 1 2022  }\\[0.5 cm]				% Course Code
	\rule{\linewidth}{0.2 mm} \\[0.4 cm]
	{ \huge \bfseries \thetitle}\\
	\rule{\linewidth}{0.2 mm} \\[1.5 cm]
	
	\begin{minipage}{0.4\textwidth}
		\begin{flushleft} \large
		%	\emph{Submitted To:}\\
		%	Name\\
          % Affiliation\\
           %contact info\\
			\end{flushleft}
			\end{minipage}~
			\begin{minipage}{0.4\textwidth}
            
			\begin{flushright} \large
			\emph{Submitted By :} \\
			Jackie Lee  
		\end{flushright}
           
	\end{minipage}\\[2 cm]
	
%	\includegraphics[scale = 0.5]{PICMathLogo.png}
    
    
    
    
	
\end{titlepage}

%%%%%%%%%%%%%%%%%%%%%%%%%%%%%%%%%%%%%%%%%%%%%%%%%%%%%%%%%%%%%%%%%%%%%%%%%%%%%%%%%%%%%%%%%

\tableofcontents
\pagebreak

%%%%%%%%%%%%%%%%%%%%%%%%%%%%%%%%%%%%%%%%%%%%%%%%%%%%%%%%%%%%%%%%%%%%%%%%%%%%%%%%%%%%%%%%%
\renewcommand{\thesection}{\arabic{section}}
\section{Introduction}
 

In the first part of the lab we get experience using Laplace transforms and partial fraction expansion to get the step response of a differential equation. We were able to utilize Python's plotting capabilities to compare the hand calculated equations to Python's built in step response and residue functions. For the 2nd part we implement the cosine method and Python's built in sig.step() function to find the time domain response of a complicated differential equation. 

\section{Equations}

Part 1:
\begin{equation*}
y"(t)+10y'(t)+24y(t)=x"(t)+6x'(t)+12x(t)
\end{equation*}

\begin{equation*}
H(s)  = \frac {s^2 + 6s + 12} { s^2 + 10s + 24}
\end{equation*}

\begin{equation*}
y(t) =  (e^{-6t} - 0.5e^{-4t} + 0.5)u(t)
\end{equation*}

Part 2:
\begin{equation*}
y^{(5)}(t)+18y^{(4)}(t)+218y^{(3)}(t)+2036y^{(2)}(t)+9085y^{(1)}(t)+25250y(t) = 25250x(t)
\end{equation*}



\section{Methodology}

For Part 1, I was able to verify that my prelab was correct using a calculator. I was able to plot its step response utilizing my result from my prelab and my defined step function. 


\begin{lstlisting}[language=Python]
#%% Code that implemented the step response from prelab and Python's sig.step function
stepresponse = (np.e**(-6*t)-0.5*np.e**(-4*t)+0.5)*step(t)

num = [1, 6, 12]
den = [1, 10, 24]

tout, yout = sig.step((num, den), T = t) #Step Response
\end{lstlisting}
Part 2 required more time than I would have thought. Getting the residue function to work was straightforward but the cosine method itself was not. I found myself having to break it down into many variables for me to have a general idea of implementing it without any compilation errors. I also had to look up multiple functions that I thought I could use. When I couldn't get the exact result, I consulted for help and to my surprise, I was fairly close and was able to finalize my code to utilize the cosine method symbolically. I was able to compare my result with Python's sig.step function by plotting both of them as subplots. 
\begin{lstlisting}[language=Python]
#%% Code that implemented the cosine method and  using Python's built in step function
newt = np.arange(0, 4.5, steps)
R2, P2, K2 = sig.residue([0,0,0,0,0,0,25250], [1,18,218,2036,9085,25250,0])

print(R2)
print(P2)
print(K2)


tdr = 0 #Initial Value as we're doing a sum from a loop
for i in range(len(R2)):
    w = np.imag(P2[i]) #return imaginary part 
    kabs = np.abs(R2[i]) #returns the magnitude of complex number
    krad = np.angle(R2[i]) #returns radians for k
    alpha = np.real(P2[i]) #only need real part of complex number
    eAlphat = np.e**(alpha*newt) ##e to the alpha t
    cos = np.cos((w * newt) + krad )
    tdr += kabs*eAlphat*cos*step(newt)
    
TempTrans2 = ([25250], [1,18,218,2036,9085,25250]) #System to use step function on

t, step2 = sig.step(TempTrans2)
\end{lstlisting}
\section{Results}

Overall, the plots that resulted in the end matched. I was able to compare my plots with other students as well so I was confident that I utilized Python's built in functions correctly.   
\\ \includegraphics[scale=0.6]{Step Response.png}

\\ \includegraphics[scale=0.6]{Time Domain Response.png}
\newpage

Partial Fraction Expansion Results for Part 1 and Part 2:
\\ \includegraphics[scale = 1.0 ]{Expansion.png}



\section{Error Analysis}

The difficult part of the lab was trying to implement the cosine method symbolically. When I initially attempted this, I tried to utilize as few lines of code as possible but it made the problem harder so I broke it down into parts. There are definitely a few things I would have not realized on my own such as not using the absolute vs using abs function. Asking for help was the final step for being to make it work and I'm glad that I did.  

\section{Questions}
1. The cosine method still works because of the fact that cosine(0) is just one. When it's a non-complex term, w = 0 and the resulting angle of K being 0. This leaves the final result of the cosine method just consisting of $ke^{at}$.

2. Since I already asked for help, I don't really have any questions in terms of this lab. I think some pseudocode would have helped for the cosine method or list the functions necessary for this part. However, I liked being able to spend some time breaking it down and finding out what I did wrong so I can improve for the next project 

\section{Conclusion}

By the end of the lab I was able to utilize Python's .residue and .step functions to plot the time domain response and step response of any differential equation. I was able to get experience with the cosine method where I would not have gained in class. Compared to the last lab, I was able to implement this function symbolically such that it would be useful for different differential equations. 

\newpage


\begin{thebibliography}{111}
\bibitem{ACMT}
Dennis M. Sullivan,
Signals and Systems for Electrical Engineers I,
 2018
\end{thebibliography}
\end{document}
