%%%%%%%%%%%%%%%%%%%%%%%%%%%%%%%%%%%%%%%%%%%
%					                      %
% Jackie Lee				              %
% ECE 351 Section 51			          %
% Lab 10				                  %
% Due April 5 2022			              %
% Latex Code for Lab 10 Report		      %
% Utilized Template on Canvas		      %
%%%%%%%%%%%%%%%%%%%%%%%%%%%%%%%%%%%%%%%%%%%

%%%%%%%%%%%%%%%%%%%%%%%%%%%%%%%%%%%%%%%%%%%
%%% DOCUMENT PREAMBLE %%%
\documentclass[12pt]{report}
\usepackage[english]{babel}
%\usepackage{natbib}
\usepackage{url}
\usepackage[utf8x]{inputenc}
\usepackage{amsmath}
\usepackage{graphicx}
\graphicspath{{images/}}
\usepackage{parskip}
\usepackage{fancyhdr}
\usepackage{vmargin}
\usepackage{listings}
\usepackage{hyperref}
\usepackage{xcolor}

\definecolor{codegreen}{rgb}{0,0.6,0}
\definecolor{codegray}{rgb}{0.5,0.5,0.5}
\definecolor{codeblue}{rgb}{0,0,0.95}
\definecolor{backcolour}{rgb}{0.95,0.95,0.92}

\lstdefinestyle{mystyle}{
    backgroundcolor=\color{backcolour},   
    commentstyle=\color{codegreen},
    keywordstyle=\color{codeblue},
    numberstyle=\tiny\color{codegray},
    stringstyle=\color{codegreen},
    basicstyle=\ttfamily\footnotesize,
    breakatwhitespace=false,         
    breaklines=true,                 
    captionpos=b,                    
    keepspaces=true,                 
    numbers=left,                    
    numbersep=5pt,                  
    showspaces=false,                
    showstringspaces=false,
    showtabs=false,                  
    tabsize=2
}
 
\lstset{style=mystyle}

\setmarginsrb{3 cm}{2.5 cm}{3 cm}{2.5 cm}{1 cm}{1.5 cm}{1 cm}{1.5 cm}

\title{Lab 10: Frequency Response}								
% Title
\author{ Jackie Lee}						
% Author
\date{April 5}
% Date

\makeatletter
\let\thetitle\@title
\let\theauthor\@author
\let\thedate\@date
\makeatother

\pagestyle{fancy}
\fancyhf{}
\rhead{\theauthor}
\lhead{\thetitle}
\cfoot{\thepage}
%%%%%%%%%%%%%%%%%%%%%%%%%%%%%%%%%%%%%%%%%%%%
\begin{document}

%%%%%%%%%%%%%%%%%%%%%%%%%%%%%%%%%%%%%%%%%%%%%%%%%%%%%%%%%%%%%%%%%%%%%%%%%%%%%%%%%%%%%%%%%

\begin{titlepage}
	\centering
    \vspace*{0.5 cm}
   % \includegraphics[scale = 0.075]{bsulogo.png}\\[1.0 cm]	% University Logo
\begin{center}    \textsc{\Large   ECE 351 - Section 51 }\\[2.0 cm]	\end{center}% University Name
	\textsc{\Large April 5 2022  }\\[0.5 cm]				% Course Code
	\rule{\linewidth}{0.2 mm} \\[0.4 cm]
	{ \huge \bfseries \thetitle}\\
	\rule{\linewidth}{0.2 mm} \\[1.5 cm]
	
	\begin{minipage}{0.4\textwidth}
		\begin{flushleft} \large
		%	\emph{Submitted To:}\\
		%	Name\\
          % Affiliation\\
           %contact info\\
			\end{flushleft}
			\end{minipage}~
			\begin{minipage}{0.4\textwidth}
            
			\begin{flushright} \large
			\emph{Submitted By :} \\
			Jackie Lee  
		\end{flushright}
           
	\end{minipage}\\[2 cm]
	
%	\includegraphics[scale = 0.5]{PICMathLogo.png}
    
    
    
    
	
\end{titlepage}

%%%%%%%%%%%%%%%%%%%%%%%%%%%%%%%%%%%%%%%%%%%%%%%%%%%%%%%%%%%%%%%%%%%%%%%%%%%%%%%%%%%%%%%%%

\tableofcontents
\pagebreak

%%%%%%%%%%%%%%%%%%%%%%%%%%%%%%%%%%%%%%%%%%%%%%%%%%%%%%%%%%%%%%%%%%%%%%%%%%%%%%%%%%%%%%%%%
\renewcommand{\thesection}{\arabic{section}}
\section{Introduction}
 

In this lab we get experience with boding plots in Python and the frequency response of an RLC circuit.

\section{Equations}
Transfer Function from RLC circuit
\begin{equation*}
H(S)=\frac{s\frac{1}{RC}}{s^2+s\frac{1}{RC}+\frac{1}{LC}}
\end{equation*}

Input Signal for Part 2
\begin{equation*}
x(t) = cos(2\pi * 100t) + cos(2\pi *3024t)+sin(2\pi * 50000t)
\end{equation*}



\section{Methodology}

For Part 1, we were given the RLC circuit and its transfer function. In the prelab we found its magnitude and phase that we were then able to compare with the phase and magnitude found in Python. There was example code and functions in the handout to lead us on the right path to plot the phase and magnitudes correctly. 

Part 2, it was also straightforward since the required functions were given in order to pass the signal in through the filter and only required some experimentation to use it correctly. 

\begin{lstlisting}[language=Python]
#%% Code that implemented the phase and magnitudes that were hand calculated and Python's implentation to calculate the phase and magnitude of a transfer function. Also implements the frequency response in Hz using con.TransferFunction(). 

mH = 20*np.log10(abs((w/(R*C))/(np.sqrt(w**4+((1/(R*C))**2-(2/(L*C)))*w**2+(1/(L*C))**2))))

pH = np.rad2deg((np.pi/2) - np.arctan2((w/(R*C)),((1/(L*C))-w**2)))

num = [1/(R*C), 0]
den = [1, 1/(R*C), 1/(L*C)]

sys = sig.TransferFunction(num,den)
sysHz = con.TransferFunction(num,den)

w2, mag2, phase2 = sig.bode(sys, w)
_ = con.bode(sysHz, w, dB = True, Hz = True, deg = True, Plot = True)

\end{lstlisting}

\begin{lstlisting}[language=Python]
#%% Code that implmented the signal from the handout and utilizing sig.bilinear and sig.lfilter to pass the singal through an RLC circuit and the filter. 
fs = 1000000
stepP2 = 1/fs

t = np.arange(0, 0.01, stepP2)

xt = np.cos(np.pi*200*t) + np.cos(2*np.pi*3024*t)+np.sin(2*np.pi*50000*t)

z, p = sig.bilinear(num, den,fs)

y = sig.lfilter(z, p, xt)


\end{lstlisting}
\section{Results}

The results were matched in terms of the plots using the phase and magnitudes from Python as well as the hand calculated values. Plotting the frequency response in Hz also seemed to work out and had a similar shape which should be expected since it's still plotting the same frequency response. Part 2 also seemed to work out since we're expecting noise to be filtered out from the initial input signal. 


Hand calculated Phase and Magnitude:
\\ \includegraphics[scale = 0.6 ]{HandCalc.png}
\newpage
Python's Phase and Magnitude:
\\ \includegraphics[scale=0.6]{Mag.png}
\\ \includegraphics[scale=0.6]{Phase.png}

Phase and Magnitude with Hz scale:
\\ \includegraphics[scale=0.6]{Hz.png}

\newpage

Input Signal x(t):

 \includegraphics[scale=0.6]{Input.png}

Output Signal y(t):

 \includegraphics[scale=0.6]{Output.png}




\section{Error Analysis}

The difficult part was getting the prelab correct, I basically had to recall the correct procedure to solve for the phase and magnitude. A lot of that was clarified by talking with other students and seeing the work in the lab itself. The other part would perhaps be implemnting the functions into Python because the end result from the hand calculations are long and could result in a missing character or a compilation error. Although being able to compare it with Python's magnitude and phase as well working through the compile errors led to the correct implementation. 

\section{Questions}
1. Given the bode plot from part 1 and our plotted input signal x(t), the output is expected to be consistent because the two are opposite of each other. We know that the filter modifies the frequency bands by changing the magnitude and phase of the input signal. 

2. The bilinear function takes an analog filter transfer function as two separate arrays as well as a sample rate fs, transforms the function from the s-domain to the z-domain, and returns the numerator and denominator of the result.
Whereas, the lfilter function takes three arrays as inputs, the digital filter function and a array of inputs. The array of inputs is inputted through the filter and an array is returned which is made up of the output values from the filter

3.  A higher fs results in a more spread out plot while the contrary shows a more condensed plot. 

4. No feedback, the work shown in the lab was useful. It would have been handy to have expected output plots in the handout but overall the lab went smoothly.     


\section{Conclusion}

By the end of the lab I was able to utilize Python to find the frequency response of a RLC circuit. It was good review on working with phases and magnitudes and Python also shows how convenient it is to simulate a signal passing through a filter. Working with it symbolically allows the code to be reused with different signals or different values for R, L, and C. This was also a good experience on how Python can implement bode plots with no issues. 

\newpage


\begin{thebibliography}{111}
\bibitem{ACMT}
Dennis M. Sullivan,
Signals and Systems for Electrical Engineers I,
 2018
\end{thebibliography}
\end{document}
