%%%%%%%%%%%%%%%%%%%%%%%%%%%%%%%%%%%%%%%%%%%
%					                      %
% Jackie Lee				              %
% ECE 351 Section 51			          %
% Lab 4					                  %
% Due February 15 2022			          %
% Latex Code for Lab 4 Report		      %
% Utilized Template on Canvas		      %
%%%%%%%%%%%%%%%%%%%%%%%%%%%%%%%%%%%%%%%%%%%

%%%%%%%%%%%%%%%%%%%%%%%%%%%%%%%%%%%%%%%%%%%
%%% DOCUMENT PREAMBLE %%%
\documentclass[12pt]{report}
\usepackage[english]{babel}
%\usepackage{natbib}
\usepackage{url}
\usepackage[utf8x]{inputenc}
\usepackage{amsmath}
\usepackage{graphicx}
\graphicspath{{images/}}
\usepackage{parskip}
\usepackage{fancyhdr}
\usepackage{vmargin}
\usepackage{listings}
\usepackage{hyperref}
\usepackage{xcolor}

\definecolor{codegreen}{rgb}{0,0.6,0}
\definecolor{codegray}{rgb}{0.5,0.5,0.5}
\definecolor{codeblue}{rgb}{0,0,0.95}
\definecolor{backcolour}{rgb}{0.95,0.95,0.92}

\lstdefinestyle{mystyle}{
    backgroundcolor=\color{backcolour},   
    commentstyle=\color{codegreen},
    keywordstyle=\color{codeblue},
    numberstyle=\tiny\color{codegray},
    stringstyle=\color{codegreen},
    basicstyle=\ttfamily\footnotesize,
    breakatwhitespace=false,         
    breaklines=true,                 
    captionpos=b,                    
    keepspaces=true,                 
    numbers=left,                    
    numbersep=5pt,                  
    showspaces=false,                
    showstringspaces=false,
    showtabs=false,                  
    tabsize=2
}
 
\lstset{style=mystyle}

\setmarginsrb{3 cm}{2.5 cm}{3 cm}{2.5 cm}{1 cm}{1.5 cm}{1 cm}{1.5 cm}

\title{Lab 4: System Step Response Using Convolution}								
% Title
\author{ Jackie Lee}						
% Author
\date{February 15}
% Date

\makeatletter
\let\thetitle\@title
\let\theauthor\@author
\let\thedate\@date
\makeatother

\pagestyle{fancy}
\fancyhf{}
\rhead{\theauthor}
\lhead{\thetitle}
\cfoot{\thepage}
%%%%%%%%%%%%%%%%%%%%%%%%%%%%%%%%%%%%%%%%%%%%
\begin{document}

%%%%%%%%%%%%%%%%%%%%%%%%%%%%%%%%%%%%%%%%%%%%%%%%%%%%%%%%%%%%%%%%%%%%%%%%%%%%%%%%%%%%%%%%%

\begin{titlepage}
	\centering
    \vspace*{0.5 cm}
   % \includegraphics[scale = 0.075]{bsulogo.png}\\[1.0 cm]	% University Logo
\begin{center}    \textsc{\Large   ECE 351 - Section 51 }\\[2.0 cm]	\end{center}% University Name
	\textsc{\Large February 15 2022  }\\[0.5 cm]				% Course Code
	\rule{\linewidth}{0.2 mm} \\[0.4 cm]
	{ \huge \bfseries \thetitle}\\
	\rule{\linewidth}{0.2 mm} \\[1.5 cm]
	
	\begin{minipage}{0.4\textwidth}
		\begin{flushleft} \large
		%	\emph{Submitted To:}\\
		%	Name\\
          % Affiliation\\
           %contact info\\
			\end{flushleft}
			\end{minipage}~
			\begin{minipage}{0.4\textwidth}
            
			\begin{flushright} \large
			\emph{Submitted By :} \\
			Jackie Lee  
		\end{flushright}
           
	\end{minipage}\\[2 cm]
	
%	\includegraphics[scale = 0.5]{PICMathLogo.png}
    
    
    
    
	
\end{titlepage}

%%%%%%%%%%%%%%%%%%%%%%%%%%%%%%%%%%%%%%%%%%%%%%%%%%%%%%%%%%%%%%%%%%%%%%%%%%%%%%%%%%%%%%%%%

\tableofcontents
\pagebreak

%%%%%%%%%%%%%%%%%%%%%%%%%%%%%%%%%%%%%%%%%%%%%%%%%%%%%%%%%%%%%%%%%%%%%%%%%%%%%%%%%%%%%%%%%
\renewcommand{\thesection}{\arabic{section}}
\section{Introduction}
 

In this lab we explore continue to explore convolution to find the step response of three transfer functions using knowledge from the previous lab and the convolution integral. 

\section{Equations}

The three transfer functions are defined as: 
\begin{equation*}
h_1(t) = e^{-2t}[u(t) - u(t-3)]
\end{equation*}
\begin{equation*}
h_2(t) = u(t-2) - u(t-6)
\end{equation*}
\begin{equation*}
h_3(t) = cos(w_0t)u(t)
\end{equation*}
\begin{equation*}
\int e^{-2(t-T)}[u(t-T) - u(t-T-3)]u(t)dT 
\end{equation*}
\begin{equation*}
\int u(t-T-2)u(T)dT - \int u(t-T-6)u(T)dT
\end{equation*}
\begin{equation*}
\int cos(0.5\pi (t-T))dT 
\end{equation*}


\section{Methodology}

For the first part, it was essentially reapplying the step function to create the new transfer functions. I was then able to plot them as a single plot through subplots following lab 1. 

Part 2 required the convolution of each of the transfer functions with the step function. This was done first through the convolution function that was created in the previous lab and was plotted on separate plots. It was then required to plot the convolution integral result equations to compare with plots generated by the convolution function. I had to redefine the axis for t between -10 and 10 so the two plots would exactly match as specified in the lab handout. 
\begin{lstlisting}[language=Python]
#%% Code that implemented and plotted the three transfer functions
h1 = (np.e**(-2*t))*(step(t)-step(t-3))
h2 = step(t - 2) - step(t - 6)
h3 = (np.cos(w*t))*step(t)

plt.figure(figsize=(12,8))

plt.subplot(3,1,1)
plt.plot(t,h1)
plt.title('Lab 4 Part 1: Signals')

plt.ylabel('h1(t)') 
plt.grid(True)

plt.subplot(3,1,2)
plt.plot(t,h2)
plt.ylabel('h2(t)')
plt.grid(which='both')

plt.subplot(3,1,3)
plt.plot(t,h3)
plt.grid(True)
plt.xlabel('t')
plt.ylabel('h3(t)')
plt.show()
\end{lstlisting}
Doing the convolution with the step function using our own convolution function was straightforward due to the last lab. Doing the convolution manually by hand was more difficult but after spending some time with it and seeing the results from the TA, it eventually came together and resulted in the following equations. 
\begin{lstlisting}[language=Python]
#%% Code that implemented the resulting convolution integral with the step function to be compared 
EqR1 = 0.5*(1-np.e**(-2*newt))*step(newt) - 0.5*(1-np.e**(-2*(newt-3)))*step(newt-3)
EqR2 = (newt-2)*step(newt-2) - (newt-6)*step(newt-6)
EqR3 = (1/(0.5*np.pi))*np.sin(0.5*np.pi*newt)*step(newt)
\end{lstlisting}
\section{Results}

Overall, the lab seemed to go well since the convolution plots between the user defined function and the plot of the convolution integral seemed to match. It was weird that if it was graphed after t = 10, that it would not match which was probably why the instructions specified to have it graphed from -10 to 10. I didn't utilize subplots for the convolutions because the y axis needed to be different for one of them. 
\newpage
Transfer Functions: 
\\ \includegraphics[scale=0.4]{Signals.png}

Convolutions from Function: 
\\ \includegraphics[scale=0.4]{myConv1.png}

\includegraphics[scale=0.4]{myConv2.png}

\includegraphics[scale=0.4]{myConv3.png}

Convolutions from Integral:
\begin{equation}
 0.5[1-e^{-2t}]u(t) - 0.5[1-e^{-2(t-3)}]u(t-3)
\end{equation}
\begin{equation}
(t+2)u(t-2) - (t+6)u(t-6) 
\end{equation}
\begin{equation}
(2/\pi)(sin(0.5\pi t)u(t) 
\end{equation}

\\ \includegraphics[scale=0.4]{IntConv1.png}

\includegraphics[scale=0.4]{IntConv2.png}

\includegraphics[scale=0.4]{IntConv3.png}



\section{Error Analysis}

There difficult part of the lab was trying to figure out why my function only had it correct up to t = 10 since it normally plots to 20 as the result of extending the t axis. I never found out why but it seemed to be a result of the initial functions because I tried using the built in convolution function that Python has and it resulted in the same thing. Since we were only required to plot it from -10 to 10 it did not impact the final results but it was strange. 

\section{Questions}
Only feedback would perhaps why we only plot the convolution from -10 to 10. I know from plotting from -20 to 20 that the convolution does not look correct but I'm not exactly sure why. 

\section{Conclusion}

By the end of the lab I got more experience plotting the convolution in Python and learned how to do it manually through the Convolution Integral. I was delayed in my completion because I spent more time on figuring out why the convolution was not correct past t = 10 but when I realized that it was not required to plot past that point and the convolution integral equations were shown in lab I was able to complete the lab.  

\newpage


\begin{thebibliography}{111}
\bibitem{ACMT}
Dennis M. Sullivan,
Signals and Systems for Electrical Engineers I,
 2018
\end{thebibliography}
\end{document}
