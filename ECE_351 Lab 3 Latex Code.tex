%%%%%%%%%%%%%%%%%%%%%%%%%%%%%%%%%%%%%%%%%%%
%					                      %
% Jackie Lee				              %
% ECE 351 Section 51			          %
% Lab 3					                  %
% Due February 8 2022			          %
% Latex Code for Lab 3 Report		      %
% Utilized Template on Canvas		      %
%%%%%%%%%%%%%%%%%%%%%%%%%%%%%%%%%%%%%%%%%%%

%%%%%%%%%%%%%%%%%%%%%%%%%%%%%%%%%%%%%%%%%%%
%%% DOCUMENT PREAMBLE %%%
\documentclass[12pt]{report}
\usepackage[english]{babel}
%\usepackage{natbib}
\usepackage{url}
\usepackage[utf8x]{inputenc}
\usepackage{amsmath}
\usepackage{graphicx}
\graphicspath{{images/}}
\usepackage{parskip}
\usepackage{fancyhdr}
\usepackage{vmargin}
\usepackage{listings}
\usepackage{hyperref}
\usepackage{xcolor}

\definecolor{codegreen}{rgb}{0,0.6,0}
\definecolor{codegray}{rgb}{0.5,0.5,0.5}
\definecolor{codeblue}{rgb}{0,0,0.95}
\definecolor{backcolour}{rgb}{0.95,0.95,0.92}

\lstdefinestyle{mystyle}{
    backgroundcolor=\color{backcolour},   
    commentstyle=\color{codegreen},
    keywordstyle=\color{codeblue},
    numberstyle=\tiny\color{codegray},
    stringstyle=\color{codegreen},
    basicstyle=\ttfamily\footnotesize,
    breakatwhitespace=false,         
    breaklines=true,                 
    captionpos=b,                    
    keepspaces=true,                 
    numbers=left,                    
    numbersep=5pt,                  
    showspaces=false,                
    showstringspaces=false,
    showtabs=false,                  
    tabsize=2
}
 
\lstset{style=mystyle}

\setmarginsrb{3 cm}{2.5 cm}{3 cm}{2.5 cm}{1 cm}{1.5 cm}{1 cm}{1.5 cm}

\title{Lab 3: Discrete Convolution}								
% Title
\author{ Jackie Lee}						
% Author
\date{February 8}
% Date

\makeatletter
\let\thetitle\@title
\let\theauthor\@author
\let\thedate\@date
\makeatother

\pagestyle{fancy}
\fancyhf{}
\rhead{\theauthor}
\lhead{\thetitle}
\cfoot{\thepage}
%%%%%%%%%%%%%%%%%%%%%%%%%%%%%%%%%%%%%%%%%%%%
\begin{document}

%%%%%%%%%%%%%%%%%%%%%%%%%%%%%%%%%%%%%%%%%%%%%%%%%%%%%%%%%%%%%%%%%%%%%%%%%%%%%%%%%%%%%%%%%

\begin{titlepage}
	\centering
    \vspace*{0.5 cm}
   % \includegraphics[scale = 0.075]{bsulogo.png}\\[1.0 cm]	% University Logo
\begin{center}    \textsc{\Large   ECE 351 - Section 51 }\\[2.0 cm]	\end{center}% University Name
	\textsc{\Large February 8 2022  }\\[0.5 cm]				% Course Code
	\rule{\linewidth}{0.2 mm} \\[0.4 cm]
	{ \huge \bfseries \thetitle}\\
	\rule{\linewidth}{0.2 mm} \\[1.5 cm]
	
	\begin{minipage}{0.4\textwidth}
		\begin{flushleft} \large
		%	\emph{Submitted To:}\\
		%	Name\\
          % Affiliation\\
           %contact info\\
			\end{flushleft}
			\end{minipage}~
			\begin{minipage}{0.4\textwidth}
            
			\begin{flushright} \large
			\emph{Submitted By :} \\
			Jackie Lee  
		\end{flushright}
           
	\end{minipage}\\[2 cm]
	
%	\includegraphics[scale = 0.5]{PICMathLogo.png}
    
    
    
    
	
\end{titlepage}

%%%%%%%%%%%%%%%%%%%%%%%%%%%%%%%%%%%%%%%%%%%%%%%%%%%%%%%%%%%%%%%%%%%%%%%%%%%%%%%%%%%%%%%%%

\tableofcontents
\pagebreak

%%%%%%%%%%%%%%%%%%%%%%%%%%%%%%%%%%%%%%%%%%%%%%%%%%%%%%%%%%%%%%%%%%%%%%%%%%%%%%%%%%%%%%%%%
\renewcommand{\thesection}{\arabic{section}}
\section{Introduction}
 

In this lab we explore how to plot convolutions of signals in Python. At this point in the semester, convolution was a new concept that has only been introduced in 350. By the end of the lab, we should be able to create a function that can produce the convolution of two signals that can then be plotted and also utilize Python's built in convolution function to check our plots. In this lab we are introduced to a new a library: scipy which contains the convolution function.  

\section{Equations}

There weren't really any new equations involved, the functions that were used in the lab still utilized step and ramp functions from the previous lab. However, we are creating 3 equations instead of the one. 
\begin{equation*}
u(t) = 0 | t < 0
\end{equation*}
\begin{equation*}
u(t) = 1 | t \geq 0
\end{equation*}
\begin{equation*}
r(t) = 1 | t >= 0
\end{equation*}
\begin{equation*}
r(t) = 1 | t >= 0
\end{equation*}
\begin{equation}
    f_1(t)=  u(t - 2) - u(t - 9)  
\end{equation}
\begin{equation}
    f_2(t)=  e^{-t} u(t)
\end{equation}
\begin{equation}
    f_3(t)=  r(t - 2)[u(t - 2) - u(t - 3] + r(4 - t)[u(t - 3) - u(t - 4)] 
\end{equation}

\section{Methodology}

For the first part, it was essentially redoing lab 2 but with different functions (inputs). The new part was plotting them all in one figure as subplots but there was an example that utilized how to have subplots in a figure so it worked out naturally. The following has my code that shows how all three of the functions were implemented using step and ramp functions. The only new part was perhaps using the math library for e and using exponents but the resources on Canvas showed how to implement exponents and e. 
\begin{lstlisting}[language=Python]
#%% Code that implemented functions 1-3 of task 1

def func1(t):
    y = np.zeros(t.shape)
    for i in range(len(t)):
        y =  step(t - 2) - step (t - 9) 
    return y


def func2(t):
    y = np.zeros(t.shape)
    for i in range(len(t)):
        y =  (math.e**-t)*step(t) 
    return y


def func3(t):
    y = np.zeros(t.shape)
    for i in range(len(t)):
        y =  ramp(t - 2) * (step(t - 2) - step(t - 3)) + ramp(4 - t) * (step(t - 3) - step(t - 4))
    return y
\end{lstlisting}

The second part was much more difficult, while there was pseudocode for the function it still left a bunch of questions because of the syntax of Python. Looking back, I still don't think I would have gotten it exactly right on my own when I saw the function in the lab. Perhaps it was due to the lack of integrals since in class that it is what we were used to. However, the neat part was that I was able to check my user defined function with the built in function. By the time I got to the lab, I think had a general idea of what it should have looked like but the additional adding/subtracting 1 or 2, I wouldn't have thought of. 

\begin{lstlisting}[language=Python]
newsteps = 0.5e-2
newt = np.arange(0, 20 +3*newsteps, newsteps)
def my_conv(f1,f2):
    Nf1 = len(f1) #get length of f1
    Nf2 = len(f2) #get length of f2
    F1 = np.append(f1, np.zeros((1, Nf2-1))) #combine the lengths to make them
    F2 = np.append(f2, np.zeros((1, Nf1-1))) #equal by append
    result = np.zeros(F1.shape)
    for i in range(Nf2+Nf1-2): #Range of both functions
        result[i] = 0
        for j in range(Nf1): #Go through the range of the input function
            if(i-j+1>0):
                result[i] += F1[j]*F2[i-j+1] #result is the area of the two
            else:
                #print(i,j)
                pass
    return result
\end{lstlisting}
\section{Results}

Overall, the lab seemed to go well since the convolution plots between the user defined and built in functions seemed to match. The last two signals made it difficult to do the convolution by hand since in lecture we have only been with the step and delta functions. I also checked my plots with other peers so I felt like results were expected. In terms of Part 1, it was a difficult to hand plot the function 3 so I did not have a clear way to verify that except by checking with other people and knowing that my step and ramp functions were verified in lab 2. 

\includegraphics[scale=0.6]{Part 1 Task 2.png}
  
\includegraphics[scale=0.6]{Part 2 Task 2.png}

\includegraphics[scale=0.6]{Part 2 Task 3.png}

\includegraphics[scale=0.6]{Part 2 Task 4.png}



\section{Error Analysis}

There only difficult part of the lab was the convolution function because no one really seemed to understand convolution as a concept. I had to try to research convolution over the weekend to have a general idea on what it was at the basic level. I wouldn't say that the textbook helped that much in terms of getting the function to work. It just seemed in lecture that we solve it graphically with the idea of flipping one function and slowly sliding it over the other function. It looked like the end result would have needed assistance in order to have a fully working function which the Youtube video that was mentioned in lab helped understand why the code works. 

\section{Questions}

1. Initially I worked alone since I attempted it over the weekend. Part 1 came naturally from Lab 2 but the convolution had me reviewing convolution online and through the textbook. It gave me rough idea of how the pseudocode would work but it wasn't until TA worked through the code with the section that it was fully operational. 
\pagebreak

2. Convolution function was the most difficult part and my problem solving involved reviewing material on convolution and comparing between the built in function and my own. The TA's lecture on convolution also helped clear a few things. I did do some sanity checks to make sure that I was right such as how I initially had the convolution go up to 100 which did not make sense but it was because I forgot to multiply by the steps so it went back down to 1. 

3. I mainly approached it from a analytical approach based on what we were given (pseudocode) and that the main debugging was working through the arrays. I think if it involved the convolution of two rectangle signals or easy convolutions I could see how a graphical approach would work. I also was checking my work by comparing to the built in function graphically so in the end I seemed to utilize both approaches to some degree.  

4. I wasn't sure if I needed to include the plots of the convolutions using the built in functions. It seemed redundant to put them on the report since it is going to produce the same plot. The equations section also made me unsure if I have to put the same equations from the previous labs or just new ones. I decided to put the old ones anyway since they were key in producing plotting the new functions.  I wonder if students were supposed to be able to create their defined convolution function on their own. Such as if it was an exam, would anyone be able to create that function? I definitely appreciate the code that was shown in lab and the quick lecture about it, it just doesn't feel that I successfully did that task. 

\section{Conclusion}

By the end of the lab I learned a bit more on convolution but more so on how to implement convolution in Python. I would say the that pseudocode helped me get started. I think it would have helped a lot of students if the video was mentioned with the pseudocode to help them understand process a bit better. It might just be how we were limited in solving convolutions as well since it was different than how they were solved in class. Maybe having the students do the convolution of two functions they already know from lecture/textbook would help. Although they could technically do it on their own. For future labs I learned that I should perform sanity checks for my plots since my initial convolution plots had too high of an output despite their functions only reaching peaks of 1.  

\newpage


\begin{thebibliography}{111}
\bibitem{ACMT}
Dennis M. Sullivan,
Signals and Systems for Electrical Engineers I,
 2018
\end{thebibliography}
\end{document}
